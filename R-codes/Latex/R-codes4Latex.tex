\documentclass[]{article}
\usepackage{lmodern}
\usepackage{amssymb,amsmath}
\usepackage{ifxetex,ifluatex}
\usepackage{fixltx2e} % provides \textsubscript
\ifnum 0\ifxetex 1\fi\ifluatex 1\fi=0 % if pdftex
  \usepackage[T1]{fontenc}
  \usepackage[utf8]{inputenc}
\else % if luatex or xelatex
  \ifxetex
    \usepackage{mathspec}
  \else
    \usepackage{fontspec}
  \fi
  \defaultfontfeatures{Ligatures=TeX,Scale=MatchLowercase}
\fi
% use upquote if available, for straight quotes in verbatim environments
\IfFileExists{upquote.sty}{\usepackage{upquote}}{}
% use microtype if available
\IfFileExists{microtype.sty}{%
\usepackage{microtype}
\UseMicrotypeSet[protrusion]{basicmath} % disable protrusion for tt fonts
}{}
\usepackage[margin=1in]{geometry}
\usepackage{hyperref}
\hypersetup{unicode=true,
            pdftitle={R-Codes},
            pdfauthor={Junhuang Xue},
            pdfborder={0 0 0},
            breaklinks=true}
\urlstyle{same}  % don't use monospace font for urls
\usepackage{color}
\usepackage{fancyvrb}
\newcommand{\VerbBar}{|}
\newcommand{\VERB}{\Verb[commandchars=\\\{\}]}
\DefineVerbatimEnvironment{Highlighting}{Verbatim}{commandchars=\\\{\}}
% Add ',fontsize=\small' for more characters per line
\usepackage{framed}
\definecolor{shadecolor}{RGB}{248,248,248}
\newenvironment{Shaded}{\begin{snugshade}}{\end{snugshade}}
\newcommand{\KeywordTok}[1]{\textcolor[rgb]{0.13,0.29,0.53}{\textbf{{#1}}}}
\newcommand{\DataTypeTok}[1]{\textcolor[rgb]{0.13,0.29,0.53}{{#1}}}
\newcommand{\DecValTok}[1]{\textcolor[rgb]{0.00,0.00,0.81}{{#1}}}
\newcommand{\BaseNTok}[1]{\textcolor[rgb]{0.00,0.00,0.81}{{#1}}}
\newcommand{\FloatTok}[1]{\textcolor[rgb]{0.00,0.00,0.81}{{#1}}}
\newcommand{\ConstantTok}[1]{\textcolor[rgb]{0.00,0.00,0.00}{{#1}}}
\newcommand{\CharTok}[1]{\textcolor[rgb]{0.31,0.60,0.02}{{#1}}}
\newcommand{\SpecialCharTok}[1]{\textcolor[rgb]{0.00,0.00,0.00}{{#1}}}
\newcommand{\StringTok}[1]{\textcolor[rgb]{0.31,0.60,0.02}{{#1}}}
\newcommand{\VerbatimStringTok}[1]{\textcolor[rgb]{0.31,0.60,0.02}{{#1}}}
\newcommand{\SpecialStringTok}[1]{\textcolor[rgb]{0.31,0.60,0.02}{{#1}}}
\newcommand{\ImportTok}[1]{{#1}}
\newcommand{\CommentTok}[1]{\textcolor[rgb]{0.56,0.35,0.01}{\textit{{#1}}}}
\newcommand{\DocumentationTok}[1]{\textcolor[rgb]{0.56,0.35,0.01}{\textbf{\textit{{#1}}}}}
\newcommand{\AnnotationTok}[1]{\textcolor[rgb]{0.56,0.35,0.01}{\textbf{\textit{{#1}}}}}
\newcommand{\CommentVarTok}[1]{\textcolor[rgb]{0.56,0.35,0.01}{\textbf{\textit{{#1}}}}}
\newcommand{\OtherTok}[1]{\textcolor[rgb]{0.56,0.35,0.01}{{#1}}}
\newcommand{\FunctionTok}[1]{\textcolor[rgb]{0.00,0.00,0.00}{{#1}}}
\newcommand{\VariableTok}[1]{\textcolor[rgb]{0.00,0.00,0.00}{{#1}}}
\newcommand{\ControlFlowTok}[1]{\textcolor[rgb]{0.13,0.29,0.53}{\textbf{{#1}}}}
\newcommand{\OperatorTok}[1]{\textcolor[rgb]{0.81,0.36,0.00}{\textbf{{#1}}}}
\newcommand{\BuiltInTok}[1]{{#1}}
\newcommand{\ExtensionTok}[1]{{#1}}
\newcommand{\PreprocessorTok}[1]{\textcolor[rgb]{0.56,0.35,0.01}{\textit{{#1}}}}
\newcommand{\AttributeTok}[1]{\textcolor[rgb]{0.77,0.63,0.00}{{#1}}}
\newcommand{\RegionMarkerTok}[1]{{#1}}
\newcommand{\InformationTok}[1]{\textcolor[rgb]{0.56,0.35,0.01}{\textbf{\textit{{#1}}}}}
\newcommand{\WarningTok}[1]{\textcolor[rgb]{0.56,0.35,0.01}{\textbf{\textit{{#1}}}}}
\newcommand{\AlertTok}[1]{\textcolor[rgb]{0.94,0.16,0.16}{{#1}}}
\newcommand{\ErrorTok}[1]{\textcolor[rgb]{0.64,0.00,0.00}{\textbf{{#1}}}}
\newcommand{\NormalTok}[1]{{#1}}
\usepackage{graphicx,grffile}
\makeatletter
\def\maxwidth{\ifdim\Gin@nat@width>\linewidth\linewidth\else\Gin@nat@width\fi}
\def\maxheight{\ifdim\Gin@nat@height>\textheight\textheight\else\Gin@nat@height\fi}
\makeatother
% Scale images if necessary, so that they will not overflow the page
% margins by default, and it is still possible to overwrite the defaults
% using explicit options in \includegraphics[width, height, ...]{}
\setkeys{Gin}{width=\maxwidth,height=\maxheight,keepaspectratio}
\IfFileExists{parskip.sty}{%
\usepackage{parskip}
}{% else
\setlength{\parindent}{0pt}
\setlength{\parskip}{6pt plus 2pt minus 1pt}
}
\setlength{\emergencystretch}{3em}  % prevent overfull lines
\providecommand{\tightlist}{%
  \setlength{\itemsep}{0pt}\setlength{\parskip}{0pt}}
\setcounter{secnumdepth}{0}
% Redefines (sub)paragraphs to behave more like sections
\ifx\paragraph\undefined\else
\let\oldparagraph\paragraph
\renewcommand{\paragraph}[1]{\oldparagraph{#1}\mbox{}}
\fi
\ifx\subparagraph\undefined\else
\let\oldsubparagraph\subparagraph
\renewcommand{\subparagraph}[1]{\oldsubparagraph{#1}\mbox{}}
\fi

%%% Use protect on footnotes to avoid problems with footnotes in titles
\let\rmarkdownfootnote\footnote%
\def\footnote{\protect\rmarkdownfootnote}

%%% Change title format to be more compact
\usepackage{titling}

% Create subtitle command for use in maketitle
\newcommand{\subtitle}[1]{
  \posttitle{
    \begin{center}\large#1\end{center}
    }
}

\setlength{\droptitle}{-2em}
  \title{R-Codes}
  \pretitle{\vspace{\droptitle}\centering\huge}
  \posttitle{\par}
  \author{Junhuang Xue}
  \preauthor{\centering\large\emph}
  \postauthor{\par}
  \predate{\centering\large\emph}
  \postdate{\par}
  \date{\today}


\begin{document}
\maketitle

\subsection{R Markdown}\label{r-markdown}

This is an R Markdown document. Markdown is a simple formatting syntax
for authoring HTML, PDF, and MS Word documents. For more details on
using R Markdown see \url{http://rmarkdown.rstudio.com}.

When you click the \textbf{Knit} button a document will be generated
that includes both content as well as the output of any embedded R code
chunks within the document. You can embed an R code chunk like this:

penisssssssssssssssssssssssssss

\begin{Shaded}
\begin{Highlighting}[]
\KeywordTok{summary}\NormalTok{(cars)}
\end{Highlighting}
\end{Shaded}

\begin{verbatim}
##      speed           dist       
##  Min.   : 4.0   Min.   :  2.00  
##  1st Qu.:12.0   1st Qu.: 26.00  
##  Median :15.0   Median : 36.00  
##  Mean   :15.4   Mean   : 42.98  
##  3rd Qu.:19.0   3rd Qu.: 56.00  
##  Max.   :25.0   Max.   :120.00
\end{verbatim}

\subsection{Including Plots}\label{including-plots}

You can also embed plots, for example:

\begin{Shaded}
\begin{Highlighting}[]
\NormalTok{##########################################################################}
\CommentTok{#                                                                        #}
\CommentTok{# Simulation                                                             #}
\CommentTok{#                                                                        #}
\NormalTok{##########################################################################}
\CommentTok{# general hints cor coding:}
\CommentTok{# monitor time each chunk code took, }
\CommentTok{# dont have to optimise chunks that used insignificant amount of time }
\CommentTok{# use for loop isf it doesnt take too long, }
\CommentTok{# use interaction function}
\CommentTok{# debug and testing codes are contained within an empty  functions}
\CommentTok{#}
\CommentTok{# V1: rand.day.time}
\CommentTok{#     simdata (sim.number, time.start, time.end, cat1n, cat2n)}
\CommentTok{#     check simulation with cumulative plot and count plot}
\CommentTok{#}
\CommentTok{# V2: Fixed the bug that caused as.POSIXct to outout different time zone}
\CommentTok{#     simdata (cat2.val:Able to add custome matrix as terminal roots)}
\CommentTok{#}
\CommentTok{# V3: rand.day}
\CommentTok{#     simdata(added leaf variable, various debugging, cleanning)}
\CommentTok{#     contain test and debug codes in an empty function}
\CommentTok{#     cumdata}
\CommentTok{#     tabulatedata }
\CommentTok{#}
\CommentTok{# Todo}
\CommentTok{# V4: fig bug in tabulatedata }
\CommentTok{#     anomaly}
\CommentTok{#     }
\CommentTok{#packages}
\CommentTok{#qpcr}
\CommentTok{#minpack.lm}
\NormalTok{###################################################################################################}

\NormalTok{### Function 1 ###}

\NormalTok{###Create a number of random times}

\CommentTok{# rand.day.time originally by Dirk Eddelbuettel 2012}
\CommentTok{# Debugged by Thomas Lumley on 28 Oct 2018}
\CommentTok{# https://stackoverflow.com/questions/14720983/efficiently-generate-a-random-sample-of-times-and-dates-between-two-dates}

\NormalTok{rand.day.time <-}\StringTok{ }\NormalTok{function(N, }\DataTypeTok{st=}\StringTok{"2006/01/01 00:00:01"}\NormalTok{, }\DataTypeTok{et=}\StringTok{"2018/12/31 23:59:59"}\NormalTok{) \{}
         \NormalTok{st <-}\StringTok{ }\KeywordTok{as.POSIXct}\NormalTok{(}\KeywordTok{strptime}\NormalTok{(st, }\DataTypeTok{format=}\StringTok{"%Y/%m/%d %H:%M:%S"}\NormalTok{, }\DataTypeTok{tz=}\StringTok{"Pacific/Auckland"}\NormalTok{))}
         \NormalTok{et <-}\StringTok{ }\KeywordTok{as.POSIXct}\NormalTok{(}\KeywordTok{strptime}\NormalTok{(et, }\DataTypeTok{format=}\StringTok{"%Y/%m/%d %H:%M:%S"}\NormalTok{, }\DataTypeTok{tz=}\StringTok{"Pacific/Auckland"}\NormalTok{))}
       \NormalTok{dt <-}\StringTok{ }\KeywordTok{as.numeric}\NormalTok{(}\KeywordTok{difftime}\NormalTok{(et, st, }\DataTypeTok{unit=}\StringTok{"sec"}\NormalTok{))}
       \NormalTok{ev <-}\StringTok{ }\KeywordTok{sort}\NormalTok{(}\KeywordTok{runif}\NormalTok{(N, }\DecValTok{0}\NormalTok{, dt))}
       \NormalTok{rt <-}\StringTok{ }\NormalTok{st +}\StringTok{ }\NormalTok{ev}
\NormalTok{\}}

\NormalTok{### testing}
\NormalTok{rand.day.time.test <-}\StringTok{ }\NormalTok{function()\{}

\NormalTok{time.start =}\StringTok{ "2006/01/01 00:00:01"}
\NormalTok{time.end =}\StringTok{ "2018/12/31 23:59:59"}
\KeywordTok{print}\NormalTok{(}\KeywordTok{rand.day.time}\NormalTok{(}\DecValTok{5}\NormalTok{,time.start,time.end)) }

\CommentTok{# system.time(rand.day.time(1000))}
\CommentTok{# system.time(rand.day.time(1000000))}

\NormalTok{test0 =}\StringTok{ }\KeywordTok{rand.day.time} \NormalTok{(}\DecValTok{100000}\NormalTok{, time.start, time.end)}
\KeywordTok{head}\NormalTok{(test0)}
\KeywordTok{tail}\NormalTok{(test0)}

\NormalTok{t.char =}\StringTok{ }\KeywordTok{as.character}\NormalTok{(test0)                           }\CommentTok{# it is taking a while}
\NormalTok{t.day =}\StringTok{ }\KeywordTok{as.character}\NormalTok{(}\KeywordTok{as.Date}\NormalTok{(t.char, }\DataTypeTok{units =} \StringTok{"days"}\NormalTok{))  }\CommentTok{# it is taking a while}
\NormalTok{dailycount =}\StringTok{ }\KeywordTok{tabulate}\NormalTok{(}\KeywordTok{as.factor}\NormalTok{(t.day))}

\KeywordTok{plot}\NormalTok{(dailycount,}\DataTypeTok{type=}\StringTok{"l"}\NormalTok{, }\DataTypeTok{col=}\StringTok{"blue"}\NormalTok{)}
\KeywordTok{abline}\NormalTok{(}\DataTypeTok{h=}\KeywordTok{mean}\NormalTok{(dailycount), }\DataTypeTok{col=}\StringTok{"red"}\NormalTok{, }\DataTypeTok{lwd=}\DecValTok{2}\NormalTok{)}
\KeywordTok{points}\NormalTok{(dailycount[}\DecValTok{1}\NormalTok{], }\DataTypeTok{pch=}\DecValTok{19}\NormalTok{, }\DataTypeTok{col=}\StringTok{"red"}\NormalTok{)}
\KeywordTok{points}\NormalTok{(}\KeywordTok{length}\NormalTok{(dailycount), dailycount[}\KeywordTok{length}\NormalTok{(dailycount)], }\DataTypeTok{pch=}\DecValTok{19}\NormalTok{, }\DataTypeTok{col=}\StringTok{"red"}\NormalTok{)}

\NormalTok{\}}
\NormalTok{##########################################################################################################}

\NormalTok{### Function 2 ###}

\NormalTok{### Create random day }

\NormalTok{rand.day <-}\StringTok{ }\NormalTok{function(}\DataTypeTok{N =} \DecValTok{1}\NormalTok{, }\DataTypeTok{st =} \StringTok{"2006/01/01 00:00:01"}\NormalTok{, }\DataTypeTok{et =} \StringTok{"2018/12/31 23:59:59"}\NormalTok{)\{}
\NormalTok{day =}\StringTok{ }\KeywordTok{as.character}\NormalTok{(}\KeywordTok{as.Date}\NormalTok{(}\KeywordTok{rand.day.time}\NormalTok{(N, st, et)), }\DataTypeTok{units =} \StringTok{"days"}\NormalTok{)}
\KeywordTok{return}\NormalTok{(day)}
\NormalTok{\}}

\NormalTok{### testing }
\NormalTok{rand.day.test <-}\StringTok{ }\NormalTok{function()\{}
  
\KeywordTok{rand.day}\NormalTok{()}
\KeywordTok{rand.day}\NormalTok{(}\DecValTok{1}\NormalTok{)}
\KeywordTok{str}\NormalTok{(}\KeywordTok{rand.day}\NormalTok{(}\DecValTok{1}\NormalTok{))}

\KeywordTok{rand.day}\NormalTok{(}\DecValTok{2}\NormalTok{)}
\KeywordTok{length}\NormalTok{(}\KeywordTok{rand.day}\NormalTok{(}\DecValTok{2}\NormalTok{))  }

\NormalTok{\}}
\NormalTok{############################################################################################################}

\NormalTok{### Function 3 ###}

\NormalTok{### Function to simulate raw data}

\CommentTok{# 1,000 to 1,000,000 simulation recommended }
\CommentTok{# time consuming parts}
   \CommentTok{# create id}
   \CommentTok{# as.character(as.date(time))}
   \CommentTok{# ifelse to assign leaf dummies}

\NormalTok{###################################################}

\NormalTok{### codes for debugging simdata}
\NormalTok{simdata.debug <-}\StringTok{ }\NormalTok{function()\{}
  
\NormalTok{sim.number =}\StringTok{ }\DecValTok{1000000}
\NormalTok{time.start =}\StringTok{ "2006/01/01 00:00:01"}
\NormalTok{time.end =}\StringTok{ "2018/12/31 23:59:59"}

\NormalTok{cat2.val =}\StringTok{ }\NormalTok{qpcR:::}\KeywordTok{cbind.na}\NormalTok{(             }
  \KeywordTok{c}\NormalTok{(}\DecValTok{2}\NormalTok{,}\DecValTok{2}\NormalTok{),                       }
  \KeywordTok{c}\NormalTok{(}\DecValTok{2}\NormalTok{,}\DecValTok{2}\NormalTok{))}

\NormalTok{cat2.val =}\StringTok{ }\NormalTok{qpcR:::}\KeywordTok{cbind.na}\NormalTok{(             }
  \KeywordTok{c}\NormalTok{(}\DecValTok{2}\NormalTok{,}\DecValTok{2}\NormalTok{),                       }
  \KeywordTok{c}\NormalTok{(}\DecValTok{2}\NormalTok{))}

\NormalTok{cat2.val =}\StringTok{ }\NormalTok{qpcR:::}\KeywordTok{cbind.na}\NormalTok{(             }
  \KeywordTok{c}\NormalTok{(}\DecValTok{200}\NormalTok{,}\DecValTok{200}\NormalTok{,}\DecValTok{100}\NormalTok{),                       }
  \KeywordTok{c}\NormalTok{(}\DecValTok{200}\NormalTok{,}\DecValTok{200}\NormalTok{,  }\DecValTok{0}\NormalTok{),}
  \KeywordTok{c}\NormalTok{(}\DecValTok{100}\NormalTok{,  }\DecValTok{0}\NormalTok{,  }\DecValTok{0}\NormalTok{))}

\NormalTok{cat2.val}
\KeywordTok{str}\NormalTok{(cat2.val)}
\NormalTok{\}}
\NormalTok{###################################################}

\NormalTok{simdata <-function(}\DataTypeTok{sim.number =} \DecValTok{1000000}\NormalTok{,                   }\CommentTok{# number of simulations}
                   \DataTypeTok{repeats =}\DecValTok{1}\NormalTok{,                             }\CommentTok{# how many time we run simulation and take mean  ##########}
                                                           \CommentTok{# how to do this????}
                   \DataTypeTok{time.start =} \StringTok{"2006/01/01 00:00:01"}\NormalTok{,     }\CommentTok{# start and end of period }
                   \DataTypeTok{time.end =} \StringTok{"2018/12/31 23:59:59"}\NormalTok{,}
                   \DataTypeTok{cat2.val =} \NormalTok{qpcR:::}\KeywordTok{cbind.na}\NormalTok{(             }\CommentTok{# number in leafs, sum to 1000}
                     \KeywordTok{c}\NormalTok{(}\DecValTok{250}\NormalTok{,}\DecValTok{250}\NormalTok{),                           }\CommentTok{# Use 0 for empty values in matrix }
                     \KeywordTok{c}\NormalTok{(}\DecValTok{250}\NormalTok{,}\DecValTok{250}\NormalTok{)))\{                         }
  
\NormalTok{###error checking}
  
  \NormalTok{if (sim.number >}\StringTok{ }\DecValTok{10000000}\NormalTok{)}
    \KeywordTok{stop}\NormalTok{(}\StringTok{"too many simulations, may be too slow"}\NormalTok{)}
  \NormalTok{if (}\KeywordTok{any}\NormalTok{(}\KeywordTok{is.na}\NormalTok{(}\KeywordTok{as.vector}\NormalTok{(cat2.val))) ==}\StringTok{ }\NormalTok{T)}
    \KeywordTok{stop}\NormalTok{(}\StringTok{"cat2.val should not contain NA, replace NA with 0"}\NormalTok{)}

\CommentTok{#  x=as.vector(cat2.val)}
\CommentTok{#  if (!is.numeric(x) || !all(is.finite(x) || x < 0))}
\CommentTok{#    stop("invalid matrix values, use real numbers")}
    
\NormalTok{### identifiers/key variables}
  
  \NormalTok{index =}\StringTok{ }\DecValTok{1}\NormalTok{:sim.number }
  \NormalTok{EventID =}\StringTok{ }\KeywordTok{sprintf}\NormalTok{(}\StringTok{"%08d"}\NormalTok{, }\KeywordTok{sample}\NormalTok{(}\DecValTok{1}\NormalTok{:}\KeywordTok{paste}\NormalTok{(}\KeywordTok{rep}\NormalTok{(}\DecValTok{9}\NormalTok{,}\DecValTok{8}\NormalTok{), }\DataTypeTok{collapse=}\StringTok{""}\NormalTok{), }
                    \NormalTok{sim.number,}\DataTypeTok{replace=}\NormalTok{F))}
 
\NormalTok{### select random times}
\CommentTok{#}
\NormalTok{rand.day.time  <-}\StringTok{ }\NormalTok{function(N, }\DataTypeTok{st=}\NormalTok{time.start, }\DataTypeTok{et =} \NormalTok{time.end) \{}
    \NormalTok{st <-}\StringTok{ }\KeywordTok{as.POSIXct}\NormalTok{(}\KeywordTok{strptime}\NormalTok{(st, }\DataTypeTok{format=}\StringTok{"%Y/%m/%d %H:%M:%S"}\NormalTok{, }\DataTypeTok{tz=}\StringTok{"Pacific/Auckland"}\NormalTok{))}
    \NormalTok{et <-}\StringTok{ }\KeywordTok{as.POSIXct}\NormalTok{(}\KeywordTok{strptime}\NormalTok{(et, }\DataTypeTok{format=}\StringTok{"%Y/%m/%d %H:%M:%S"}\NormalTok{, }\DataTypeTok{tz=}\StringTok{"Pacific/Auckland"}\NormalTok{))}
    \NormalTok{dt <-}\StringTok{ }\KeywordTok{as.numeric}\NormalTok{(}\KeywordTok{difftime}\NormalTok{(et, st, }\DataTypeTok{unit=}\StringTok{"sec"}\NormalTok{))}
    \NormalTok{ev <-}\StringTok{ }\KeywordTok{sort}\NormalTok{(}\KeywordTok{runif}\NormalTok{(N, }\DecValTok{0}\NormalTok{, dt)) }\CommentTok{#<------------------------------- add seasonality here???}
    \NormalTok{rt <-}\StringTok{ }\NormalTok{st +}\StringTok{ }\NormalTok{ev}
\NormalTok{\}}

\NormalTok{time =}\StringTok{ }\KeywordTok{rand.day.time}\NormalTok{(sim.number, time.start, time.end)}
\NormalTok{time.char =}\StringTok{ }\KeywordTok{as.character}\NormalTok{(time)}
\NormalTok{time.day =}\StringTok{ }\KeywordTok{as.character}\NormalTok{(}\KeywordTok{as.Date}\NormalTok{(time.char, }\DataTypeTok{units =} \StringTok{"days"}\NormalTok{))}

\CommentTok{#----------------------------------------------------------#}
\NormalTok{### simulate level 1 and 2 brunch using terminal root matrix}

\NormalTok{cat1.val =}\StringTok{ }\KeywordTok{colSums}\NormalTok{(cat2.val, }\DataTypeTok{na.rm=}\NormalTok{T)}
\NormalTok{cat2.tot =}\StringTok{ }\KeywordTok{sum}\NormalTok{(}\KeywordTok{colSums}\NormalTok{(!}\KeywordTok{is.na}\NormalTok{(cat2.val)))}

\NormalTok{cat1.n =}\StringTok{ }\KeywordTok{length}\NormalTok{(cat1.val)}
\NormalTok{cat2.n =}\StringTok{ }\KeywordTok{nrow}\NormalTok{(cat2.val)}
  
\NormalTok{cat1.let =}\StringTok{ }\KeywordTok{c}\NormalTok{(LETTERS[}\DecValTok{1}\NormalTok{:cat1.n])}
\NormalTok{cat2.let =}\StringTok{ }\KeywordTok{c}\NormalTok{(LETTERS[}\DecValTok{1}\NormalTok{:cat2.n])}

\NormalTok{sim.cat1 =}\StringTok{ }\KeywordTok{rep}\NormalTok{(cat1.let, cat1.val)}
\NormalTok{cat1 =}\StringTok{ }\KeywordTok{sample}\NormalTok{(sim.cat1 , sim.number, }\DataTypeTok{replace =} \OtherTok{TRUE}\NormalTok{)}
\NormalTok{cat2 =}\StringTok{ }\KeywordTok{rep}\NormalTok{(}\OtherTok{NA}\NormalTok{,sim.number)}
\KeywordTok{tabulate}\NormalTok{(}\KeywordTok{as.factor}\NormalTok{(sim.cat1))}
\KeywordTok{tabulate}\NormalTok{(}\KeywordTok{as.factor}\NormalTok{(cat1))}

\NormalTok{Category =}\StringTok{ }\KeywordTok{cbind}\NormalTok{(cat1, cat2)}
\KeywordTok{head}\NormalTok{(Category)}

\NormalTok{for(i in }\DecValTok{1}\NormalTok{:cat2.n)\{}
  \NormalTok{Category[}\KeywordTok{which}\NormalTok{(Category[,}\DecValTok{1}\NormalTok{] ==}\StringTok{ }\NormalTok{LETTERS[i]),}\DecValTok{2}\NormalTok{] =}\StringTok{ }\KeywordTok{sample}\NormalTok{(}\KeywordTok{rep}\NormalTok{(cat2.let, cat2.val[,i]), }
          \KeywordTok{sum}\NormalTok{(Category[,}\DecValTok{1}\NormalTok{] ==}\StringTok{ }\NormalTok{LETTERS[i]), }\DataTypeTok{replace =} \OtherTok{TRUE}\NormalTok{)}
\NormalTok{\}}
\KeywordTok{head}\NormalTok{(Category)}
\KeywordTok{table}\NormalTok{(Category[,}\DecValTok{1}\NormalTok{], Category[,}\DecValTok{2}\NormalTok{])}

\NormalTok{Category =}\StringTok{ }\KeywordTok{as.data.frame}\NormalTok{(Category)}
\NormalTok{Category$leaf =}\StringTok{ }\KeywordTok{paste}\NormalTok{(Category$cat1, Category$cat2, }\DataTypeTok{sep =} \StringTok{""}\NormalTok{)}
\KeywordTok{head}\NormalTok{(Category)}

\NormalTok{###assign name and value to dummy variable, name same as value}

\NormalTok{cat1.name =}\StringTok{ }\NormalTok{cat1.let}
\NormalTok{cat1.name}
\NormalTok{cat2.name =}\StringTok{ }\KeywordTok{levels}\NormalTok{(}\KeywordTok{interaction}\NormalTok{(cat1.let, cat2.let, }\DataTypeTok{sep =} \StringTok{""}\NormalTok{, }\DataTypeTok{lex.order =} \OtherTok{TRUE}\NormalTok{))}
\NormalTok{cat2.name }
\NormalTok{names =}\StringTok{ }\KeywordTok{c}\NormalTok{(cat1.name, cat2.name)}
\NormalTok{names}

\KeywordTok{substr}\NormalTok{(cat2.name,}\DecValTok{1}\NormalTok{,}\DecValTok{1}\NormalTok{)}
\KeywordTok{substr}\NormalTok{(cat2.name,}\DecValTok{2}\NormalTok{,}\DecValTok{2}\NormalTok{)}

\NormalTok{for(i in }\DecValTok{1}\NormalTok{:(cat1.n+cat1.n*cat2.n))\{}
\NormalTok{Category[,i}\DecValTok{+3}\NormalTok{] =}\StringTok{ }\KeywordTok{rep}\NormalTok{(names[i],sim.number)}
\NormalTok{\}}
\KeywordTok{colnames}\NormalTok{(Category) =}\StringTok{ }\KeywordTok{c}\NormalTok{(}\StringTok{"cat1"}\NormalTok{,}\StringTok{"cat2"}\NormalTok{,}\StringTok{"leaf"}\NormalTok{,cat1.name,cat2.name)}
\KeywordTok{head}\NormalTok{(Category)}

\NormalTok{### use ifelse to assign dummies variables}

\NormalTok{for(i in }\DecValTok{1}\NormalTok{:cat1.n)\{}
  \NormalTok{Category[,i}\DecValTok{+3}\NormalTok{] =}\StringTok{ }\KeywordTok{ifelse}\NormalTok{(Category$cat1 ==}\StringTok{ }\NormalTok{Category[,i}\DecValTok{+3}\NormalTok{], }\DecValTok{1}\NormalTok{, }\DecValTok{0}\NormalTok{)}
\NormalTok{\}}

\NormalTok{for(i in }\DecValTok{1}\NormalTok{:(cat1.n*cat2.n))\{}
  \NormalTok{Category[,i}\DecValTok{+3}\NormalTok{+cat1.n] =}\StringTok{ }\KeywordTok{ifelse}\NormalTok{(Category$cat1 ==}\StringTok{ }\KeywordTok{substr}\NormalTok{(Category[,i}\DecValTok{+3}\NormalTok{+cat1.n],}\DecValTok{1}\NormalTok{,}\DecValTok{1}\NormalTok{) &}
\StringTok{                                 }\NormalTok{Category$cat2 ==}\StringTok{ }\KeywordTok{substr}\NormalTok{(Category[,i}\DecValTok{+3}\NormalTok{+cat1.n],}\DecValTok{2}\NormalTok{,}\DecValTok{2}\NormalTok{), }\DecValTok{1}\NormalTok{, }\DecValTok{0}\NormalTok{)}
\NormalTok{\}}
\KeywordTok{head}\NormalTok{(Category)}

\NormalTok{### delete columns if column sum is 0}

\NormalTok{dummies =}\StringTok{ }\NormalTok{Category[-(}\DecValTok{1}\NormalTok{:}\DecValTok{3}\NormalTok{)]}
\NormalTok{cat2.val}
\KeywordTok{colSums}\NormalTok{(dummies)}

\NormalTok{emptyroot =}\StringTok{ }\KeywordTok{which}\NormalTok{(}\KeywordTok{colSums}\NormalTok{(dummies) ==}\StringTok{ }\DecValTok{0}\NormalTok{)}
\NormalTok{emptyroot2 =}\StringTok{ }\KeywordTok{which}\NormalTok{(}\KeywordTok{colSums}\NormalTok{(dummies) ==}\StringTok{ }\DecValTok{2}\NormalTok{) }\CommentTok{# <------------------------fix here for no col == 0}
\NormalTok{emptyroot}
\NormalTok{emptyroot2}
\NormalTok{test1 =}\StringTok{ }\NormalTok{Category[-(emptyroot}\DecValTok{+3}\NormalTok{)]}
\KeywordTok{head}\NormalTok{(test1)}
\NormalTok{test2 =}\StringTok{  }\NormalTok{Category}
\KeywordTok{head}\NormalTok{(test2)}

\NormalTok{Category =}\StringTok{ }\NormalTok{Category[-(emptyroot}\DecValTok{+3}\NormalTok{)]}
\KeywordTok{head}\NormalTok{(Category)}


\NormalTok{### create dataframe}

\NormalTok{sim.data.df =}\StringTok{ }\KeywordTok{data.frame}\NormalTok{(}\KeywordTok{cbind}\NormalTok{(index, EventID, time.char, time.day, Category))}
\NormalTok{sim.data.df$index     =}\StringTok{ }\KeywordTok{as.numeric}  \NormalTok{(sim.data.df$index)}
\NormalTok{sim.data.df$EventID   =}\StringTok{ }\KeywordTok{as.character}\NormalTok{(sim.data.df$EventID)}
\NormalTok{sim.data.df$time.char =}\StringTok{ }\KeywordTok{as.character}\NormalTok{(sim.data.df$time.char)}
\NormalTok{sim.data.df$time.day  =}\StringTok{ }\KeywordTok{as.Date}     \NormalTok{(sim.data.df$time.day)}

\KeywordTok{return}\NormalTok{(sim.data.df)}
\NormalTok{\} }


\NormalTok{###Extras}

\CommentTok{#add categorical variables by random sampling? }
\CommentTok{#add contineous variables by random sampling??? using some kind of model??}

\NormalTok{########################################################################################################3333#}

\NormalTok{###Testing }
\NormalTok{simdata.test <-}\StringTok{ }\NormalTok{function()\{}
  
\NormalTok{test1 =}\StringTok{ }\KeywordTok{simdata} \NormalTok{(}\DecValTok{1000}\NormalTok{)}
\KeywordTok{head}\NormalTok{(test1)}
\KeywordTok{str}\NormalTok{(test1)}
\KeywordTok{colSums}\NormalTok{(test1[,}\DecValTok{7}\NormalTok{:}\KeywordTok{ncol}\NormalTok{(test1)])    }

\NormalTok{v2 =}\StringTok{ }\KeywordTok{cbind}\NormalTok{(                       }\CommentTok{# number in terminal roots, sum to 1000}
  \KeywordTok{c}\NormalTok{(}\DecValTok{200}\NormalTok{,}\DecValTok{200}\NormalTok{,}\DecValTok{100}\NormalTok{),                 }
  \KeywordTok{c}\NormalTok{(}\DecValTok{200}\NormalTok{,}\DecValTok{200}\NormalTok{,  }\DecValTok{0}\NormalTok{),}
  \KeywordTok{c}\NormalTok{(}\DecValTok{100}\NormalTok{,  }\DecValTok{0}\NormalTok{,  }\DecValTok{0}\NormalTok{))}

\NormalTok{test2 =}\StringTok{ }\KeywordTok{simdata} \NormalTok{(}\DecValTok{1000}\NormalTok{,}\DataTypeTok{cat2.val =} \NormalTok{v2)}
\KeywordTok{head}\NormalTok{(test2)}
\KeywordTok{str}\NormalTok{(test2)}
\KeywordTok{colSums}\NormalTok{(test2[,}\DecValTok{7}\NormalTok{:}\KeywordTok{ncol}\NormalTok{(test2)])}

\NormalTok{test3 =}\StringTok{ }\KeywordTok{simdata} \NormalTok{(}\DecValTok{1000000}\NormalTok{,}\DataTypeTok{cat2.val =} \NormalTok{v2)}
\KeywordTok{head}\NormalTok{(test3)}
\KeywordTok{str}\NormalTok{(test3)}

\KeywordTok{system.time}\NormalTok{(}\KeywordTok{simdata} \NormalTok{(}\DecValTok{1000}\NormalTok{))}
\KeywordTok{system.time}\NormalTok{(}\KeywordTok{simdata} \NormalTok{(}\DecValTok{1000000}\NormalTok{))}

\NormalTok{\}}
\NormalTok{#########################################################################################################}

\NormalTok{###################################################################################################}

\CommentTok{# Function 5, 6}

\NormalTok{## functions to check simulation}
\NormalTok{## use ggplot 2 later if we want a good presentable plot}

\CommentTok{# create a function for tabulation}
\NormalTok{#################################################}

\CommentTok{#debug for simulation checking codes}
\NormalTok{simcheck.debug <-}\StringTok{ }\NormalTok{function()\{}
  
\NormalTok{v2 =}\StringTok{ }\KeywordTok{cbind}\NormalTok{(            }
  \KeywordTok{c}\NormalTok{(}\DecValTok{200}\NormalTok{,}\DecValTok{200}\NormalTok{,}\DecValTok{100}\NormalTok{),                 }
  \KeywordTok{c}\NormalTok{(}\DecValTok{200}\NormalTok{,}\DecValTok{200}\NormalTok{,  }\DecValTok{0}\NormalTok{),}
  \KeywordTok{c}\NormalTok{(}\DecValTok{100}\NormalTok{,  }\DecValTok{0}\NormalTok{,  }\DecValTok{0}\NormalTok{))}

\NormalTok{sample1 =}\StringTok{ }\KeywordTok{simdata} \NormalTok{(}\DecValTok{100000}\NormalTok{,}\DataTypeTok{cat2.val =} \NormalTok{v2)}
\KeywordTok{head}\NormalTok{(sample1)}
\KeywordTok{str}\NormalTok{(sample1)}
\NormalTok{x =}\StringTok{ }\NormalTok{sample1}
\NormalTok{\}}
\NormalTok{##################################################}
\end{Highlighting}
\end{Shaded}

Note that the \texttt{echo\ =\ FALSE} parameter was added to the code
chunk to prevent printing of the R code that generated the plot.


\end{document}
