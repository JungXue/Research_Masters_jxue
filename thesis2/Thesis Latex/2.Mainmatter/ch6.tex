\chapter{Conclusions}

	the size definition of point anomaly and  period anomaly used are

$	if mu_pointanomaly = 100, size of point anomaly will be 100, however for period anomaly, mu_period anomaly is the highest point, and not the sum total count through out the period. the distribution of period anomaly will influence the size of each day.  $

for example, 

List of simulated data used can be found in appendix ref

name  simulation  anomaly

6 Note 1000 sample did not give stable dic, try 10000


t.start = "2006/01/01 00:00:01"   
t.end = "2018/12/31 23:59:59"
100000

matrix1 = cbind(            
c(100,100),                 
c(100,100))

\begin{table}[ht]
	\centering
	\begin{tabular}{rrrrrrr}
		\hline
		N & st & et & \text{cat2.val}  & output file\\ 
		\hline
		100000 & "2006/01/01 00:00:01" & "2018/12/31 23:59:59" & matrix1 & raw1.df \\ 
		100000 & "2006/01/01 00:00:01" & "2018/12/31 23:59:59" & matrix2 & raw2.df \\ 
		100000 & "2006/01/01 00:00:01" & "2018/12/31 23:59:59" & matrix3 & raw3.df \\ 
		100000 & "2006/01/01 00:00:01" & "2018/12/31 23:59:59" & matrix4 & raw4.df \\ 
		\hline
	\end{tabular}
	\caption{Function settings for synthetic data used in section XXX}
\end{table}