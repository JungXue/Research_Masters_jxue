\pagenumbering{arabic}
	
\chapter{Background}
	
	\section{Introduction}
%------------------------------------------------------------------------------------------
	\section{Motivation}
	

%------------------------------------------------------------------------------------------
	\section{Hiercrchical models}
	
%http://www.texample.net/tikz/examples/merge-sort-recursion-tree/
		\paragraph{}
		
%\comm{
\begin{center}
$\begin{bmatrix}
	y_{t} \\
	y_{A,t}\\
	y_{B,t} \\
	y_{AA,t} \\
	y_{AB,t}\\
	y_{BA,t} \\
	y_{BB,t}
\end{bmatrix}$
=
$\begin{bmatrix}
	1 & 1 & 1 & 1 \\
	1 & 1 & 0 & 0 \\
	0 & 0 & 1 & 1 \\
	1  & 0  & 0  & 0  \\
	0  & 1  & 0  & 0  \\
	0  & 0  & 1  & 0  \\
	0  & 0  & 0  & 1
\end{bmatrix}$
$\begin{bmatrix}
	y_{AA,t} \\
	y_{AB,t} \\
	y_{BA,t} \\
	y_{BB,t}
\end{bmatrix}$
\end{center}
%}

\begin{equation}
{y}_t={S}{b}_{t},
\tag{add tag here}
\end{equation}
	
%------------------------------------------------------------------------------------------	
	
	\section{Bayesian Models}
	
		\section{Models}
	
	N = number of simulations\\
	Y = count of patient arrivals everyday\\
	S = Seasonality\\
	T = Trend\\
	C = cycles? \\
	A = Anomalies\\
	d = Dummies\\
	D = disease category (bottom level)\\
	$\mu_0$ = mean
	
	
	$\sum{Y} = 1,000,000$\\
	
	
	
	\begin{equation}
	Y = \mu_0 + \sum{d_iS_i}+ \sum{d_iA_i} + \sum{d_iD_i} + \sum{d_iT_i}\\
	\end{equation}
	
	$L(\theta|x)$
%------------------------------------------------------------------------------------------
	\section{Anomaly Detection}

%------------------------------------------------------------------------------------------

	\section{Outline of Thesis}
	
	jjj\\
	jjj\\
	jjj
	
	\color{red}
	
	\newpage
	\section{Notations and conventions }
	
	(add any more format that you will use, do this after most of thesis is done)
	
	\textbf{Note:} The following notation and conventions is used throughout this thesis to aid\\ readability:\\

	\begin{tabular}{rl}
	\textbf{Software} & Software is denoted by sans-serif font.\\\\
		\textbf{R functions} & Functions are denoted with monospace font with two trailing \\
		& parentheses. Arguments are similarly displayed in monospace \\
		& with format argument = value.\\\\
		\textbf{R packages} & Package names are given in boldface.\\\\
		\textbf{Computer code} &  Any code is denoted with monospace font.\\\\
	\end{tabular}

	\color{black}
