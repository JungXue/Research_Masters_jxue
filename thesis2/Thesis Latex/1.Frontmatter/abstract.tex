\chapter*{Abstract}

For datasets with a hierarchical structure, there may be important information that is contained within its hierarchical relationship, such as the aggregation constraint, that children of a category will sum to itself. With a hierarchical structure, there are different levels of hierarchy, modelling with only one level of hierarchy may miss vital information and reduce the power of our models. A hierarchical model that considers all levels could yield more promising predictions.

\newpara

We compared posterior outputs and anomaly signals using Individual Bayesian Model (IBM) and Hierarchical Bayesian Model (HBM), with several simulation studies and found that HBM seems to out-perform IBM. However, there is only a minimal difference in anomaly signals.

\newpara

ICD-9 is a standard of disease diagnosis with a hierarchical structure. We also investigate the use of HBM as a possible approach for the detection of anomalies for different rarities of disease with ICD-9 categories, and found that for common disease with a large counts it is possible to detect anomalies at higher levels, however when the disease is rare of extremely rare it is recommended to look at lower levels for more reliable anomaly signals. 

\newpara 

Hierarchical models seem to provide better predictions, but the process of building the model is complicated, and with complex hierarchical structures, it becomes a tedious task. Independent models may be overall inferior but are easier to build and could produce adequate results. 

\newpara

Use of HBM provides better results than IBM for the case of anomaly detection for hierarchical datasets, HBM tends to have better goodness of fit and the difference in detection results are apparent. HBM is a feasible design that is well-suited for datasets that have attributes with large hierarchical structures and have an advantage over IBM in many areas. 
		
		